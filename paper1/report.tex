\documentclass[sigconf]{acmart}

\usepackage{hyperref}

\usepackage{endfloat}
\renewcommand{\efloatseparator}{\mbox{}} % no new page between figures

\usepackage{booktabs} % For formal tables

\settopmatter{printacmref=false} % Removes citation information below abstract
\renewcommand\footnotetextcopyrightpermission[1]{} % removes footnote with conference information in first column
\pagestyle{plain} % removes running headers

\begin{document}
\title{Big Data Application in Web Search and Text Mining}


\author{Wenxuan Han}
% \orcid{1234-5678-9012}
\affiliation{%
  \institution{Indiana University Bloomingtonn}
  \streetaddress{1150 S Clarizz Blvd}
  \city{Bloomington} 
  \state{Indiana} 
  \postcode{47401-4294}
}
\email{wenxhan@iu.edu}

% The default list of authors is too long for headers}
% \renewcommand{\shortauthors}{B. Trovato et al.}


\begin{abstract}
Because of the rapid development of social media, there are gigantic amount of data generated in every second on the web. And those data could be stored in any forms like text, videos, images or their combinations. The more complicated forms of data, the more space it will take up and will cost more time to read it. Although most of today's personal computers have a very high performance, it is extremely difficult to process and analyze useful text information from those huge amount of unstructured data by using traditional single computer methods without the help of big data tools or text mining techniques. Fortunately, the improvements in big data application are also increasing fast in order to support those difficult works on web search and text mining. In this paper, we first study the data analytic steps in web search, then analyze some of the popular approaches or algorithms (e.g. Hubs, PageRank, etc), and at last, we discuss their applications in this field of big data.
\end{abstract}

\keywords{social media, web search, text mining, PageRank, Hubs}


\maketitle

\section{Introduction}

The \textit{proceedings} are the records of a
conference. ACM seeks to give these
conference by-products a uniform, high-quality appearance.  To do
this, ACM has some rigid requirements for the format of the
proceedings documents: there is a specified format (balanced double
columns), a specified set of fonts (Arial or Helvetica and Times
Roman) in certain specified sizes, a specified live area, centered on
the page, specified size of margins, specified column width and gutter
size \cite{editor00}.

\section{Data analytic steps}
This part in in in 


\begin{acks}

  The authors would like to thank 

\end{acks}

\bibliographystyle{ACM-Reference-Format}
\bibliography{report} 

\end{document}
