\documentclass[sigconf]{acmart}

\input{format/i523}

\begin{document}
\title{Clustering Algorithms in Big Data Analysis}


\author{Wenxuan Han}
% \orcid{1234-5678-9012}
\affiliation{%
  \institution{Indiana University Bloomington}
  \streetaddress{1150 S Clarizz Blvd}
  \city{Bloomington} 
  \state{Indiana} 
  \postcode{47401-4294}
}
\email{wenxhan@iu.edu}


% The default list of authors is too long for headers}
% \renewcommand{\shortauthors}{G. v. Laszewski}


\begin{abstract}
Data Mining is a kind of popular method which intent to extract and analyze useful information from data. However, the rapid development of computing technologies has caused the size of data sets become extremely huge. Because of the high complexity of these data sets, traditional data mining approaches or algorithms are not appropriate to be used. Therefore, it is very important to discover the similarities of data and use them to divide data into different groups. Nowadays, clustering algorithms have emerged as a powerful meta-learning tool which provided the goal to categorize data into clusters such that objects are grouped in the same cluster are similar according to specific properties like traits and behaviors. In this paper, we have introduced several types of clustering technologies in data mining with some most commonly used algorithms, and compare the advantages and disadvantages between them during the large data sets environment.
\end{abstract}

\keywords{I523, HID209, Big Data, Clustering Algorithms}


\maketitle



\section{Introduction}

Today, human's progress on Internet, Internet of Things (IoT) and other computing technologies lead to the growth of many related applications. Due to the usage of these applications in human's daily life, there are huge amount of data generated every second which let the concept of big data emerged. Since these data may reveal hidden information which is interesting and important for people to study, the management of big data plays a significant role. We viewed technologies which used to extracting meaningful and required information or to find out unseen relationship between the data as Data Mining \cite{dcar}, and clustering is the one of the main tasks for processing data.

Clustering of big data has gained popularity in the last decade due to increased demand to process of large data sets \cite{bcab}. Data clustering algorithms were developed as a meta-learning tool to analyze massive data accurately. Its main purpose is to find similarities between all objects in a given data set and categorize them into groups by specific metrics which means that clustering is a process to manage similar objects in a same group. And since the data set processed by a clustering algorithm is unlabeled, data clustering has also considered as an unsupervised learning technique \cite{dcaa}.

Because of the large amounts of data produced by many applications from different sources in today's world, the demand of efficiency clustering algorithms is increasing. However, it is difficult to create a perfect algorithm which has the capacity to satisfy requirement of all situations. Thus, depend on the purposes of clustering and the given data set, people should utilize the corresponding clustering approach \cite{bcab}. Here are the main types of clustering (algorithms) for common use:
\begin{itemize}
\item Exclusive or partition-based clustering;
\item Hierarchical clustering;
\item Density-based clustering;
\item Grid-based clustering;
\item Model-based clustering.
\end{itemize}
In the rest of this paper we will discuss these cluster types as well as some of their applications.

\section{Clustering Types}
\begin{enumerate}
\item Partition clustering \\
aaa
\item Density-based clustering
\end{enumerate}
\section{Conclusion}


\begin{acks}

The author would like to thank Professor Gregor von Laszewski and all TAs for providing the resource, tutorials and other related materials to write this paper.

\end{acks}

\bibliographystyle{ACM-Reference-Format}
\bibliography{report}

% \section{Issues}

\DONE{Well done! But be aware of the warnings of the bib!}




\end{document}
