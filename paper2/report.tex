\documentclass[sigconf]{acmart}

\usepackage{graphicx}
\usepackage{hyperref}
\usepackage{todonotes}

\usepackage{endfloat}
\renewcommand{\efloatseparator}{\mbox{}} % no new page between figures

\usepackage{booktabs} % For formal tables

\settopmatter{printacmref=false} % Removes citation information below abstract
\renewcommand\footnotetextcopyrightpermission[1]{} % removes footnote with conference information in first column
\pagestyle{plain} % removes running headers

\newcommand{\TODO}[1]{\todo[inline]{#1}}

\begin{document}
\title{Clustering Algorithms in Big Data Analysis}


\author{Wenxuan Han}
% \orcid{1234-5678-9012}
\affiliation{%
  \institution{Indiana University Bloomington}
  \streetaddress{1150 S Clarizz Blvd}
  \city{Bloomington} 
  \state{Indiana} 
  \postcode{47401-4294}
}
\email{wenxhan@iu.edu}


% The default list of authors is too long for headers}
% \renewcommand{\shortauthors}{G. v. Laszewski}


\begin{abstract}
Data Mining is a kind of popular method which intent to extract and analyze useful information from data. However, the rapid development of computing technologies has caused the size of data sets become extremely huge. Because of the high complexity of these data sets, traditional data mining approaches or algorithms are not appropriate to be used. Therefore, it is very important to discover the similarities of data and use them to divide data into different groups. Nowadays, clustering algorithms have emerged as a powerful meta-learning tool which provided the goal to categorize data into clusters such that objects are grouped in the same cluster are similar according to specific properties like traits and behaviors. In this paper, we have introduced several types of clustering technologies in data mining with some most commonly used algorithms, and compare the advantages and disadvantages between them during the large data sets environment.
\end{abstract}

\keywords{I523, HID209, Big Data, Clustering Algorithms}


\maketitle



\section{Introduction}

Today, human's progress on Internet, Internet of Things (IoT) and other computing technologies lead to the growth of many related applications. Due to the usage of these applications in human's daily life, there are huge amount of data generated every second which let the concept of big data emerged. Since these data may reveal hidden information which is interesting and important for people to study, the management of big data plays a significant role. We viewed technologies which used to extracting meaningful and required information or to find out unseen relationship between the data as Data Mining \cite{dcar}, and clustering is the one of the main tasks for processing data.

Clustering of big data has gained popularity in the last decade due to increased demand to process of large data sets \cite{bcab}. Data clustering algorithms were developed as a meta-learning tool to analyze massive data accurately. Its main purpose is to find similarities between all objects in a given data set and categorize them into groups by specific metrics which means that clustering is a process to manage similar objects in a same group. And since the data set processed by a clustering algorithm is unlabeled, data clustering has also considered as an unsupervised learning technique \cite{dcaa}.

Because of the large amounts of data produced by many applications from different sources in today's world, the demand of efficiency clustering algorithms is increasing. However, it is difficult to create a perfect algorithm which has the capacity to satisfy requirement of all situations. Thus, depend on the purposes of clustering and the given data set, people should utilize the corresponding clustering approach \cite{bcab}. Here are the main types of clustering (algorithms) which commonly used:
\begin{itemize}
\item Exclusive or partitioning-based clustering;
\item Hierarchical clustering;
\item Density-based clustering;
\item Grid-based clustering;
\item Model-based clustering.
\end{itemize}
In the rest of this paper we will discuss these cluster types as well as some of the applications.

\section{Clustering Types}
In this section, we will see the main par of clustering types. Each clustering has its only purpose to solve the spe problem.
\subsection{Exclusive or Partitioning-based Clustering}

Partitioning is a kind of technique which used to decompose the set of data objects into several non-overlapping partitions (each partition represents a cluster) such that each data object is in exactly one partition.

Given a set of $N$ data points, a partitioning method usually start with a random partitioning and refine it iteratively to find a partition of $K$ ($K \leq N$) clusters that optimizes the chosen partitioning criterion. While classifying data points into $K$ groups, it must satisfy two rules: one point must be present in each cluster and one cluster must have one set \cite{dcar}. One of the most famous and commonly used partitioning techniques is k-means.

\subsection{Hierarchical-based Clustering}

Hierarchical clustering groups a big scale of data set by creating a cluster tree or dendrogram. The tree is not a single set of clusters, but rather a multilevel hierarchy, where closeness clusters at one level of leaf nodes are joined as new clusters that compose to nodes at the next level \cite{math}. Generally, there are two approaches applied in hierarchical clustering which are agglomerative method (top-down approach) and divisive method (bottom-up approach).

For agglomerative method, it starts in two or more clusters recursively merged as the most applicable cluster by moving up the hierarchy. However, for divisive method, it starts in a single cluster and recursively split to find applicable cluster for the items by moving down the hierarchy \cite{dcar}. Although the time complexity for both approaches are huge ($O(n^2\log(n))$ for agglomerative and $O(2^n)$ for divisive) which makes them execute too slow during the big data sets, it is possible to choose optimal agglomerative methods such as SLINK for single-linkage and CLINK for complete-linkage clustering to reduce the complexity to $O(n^2)$. BIRCH, CURE, ROCK and Chameleon are some typical algorithms that applied under hierarchical-based clustering.

\subsection{Density-based Clustering}

\section{Clustering Algorithms}

Here in this section introduces two kinds of well-known and commonly used clustering algorithms: k-means algorithm which applied partitioning-based clustering and CLIQUE algorithm which applied grid-based clustering.

\subsection{K-means Algorithm}
\subsection{CLIQUE}

\section{Comparson between clustering types}

\section{Conclusion}


\begin{acks}

The author would like to thank Professor Gregor von Laszewski and all TAs for providing the resource, tutorials and other related materials to write this paper.

\end{acks}

\bibliographystyle{ACM-Reference-Format}
\bibliography{report}

% \section{Issues}

\DONE{Example of done item: Once you fix an item, change TODO to DONE}

\subsection{Assignment Submission Issues}

    \TODO{Do not make changes to your paper during grading, when your repository should be frozen.}

\subsection{Uncaught Bibliography Errors}

    \TODO{Missing bibliography file generated by JabRef}
    \TODO{Bibtex labels cannot have any spaces, \_ or \& in it}
    \TODO{Citations in text showing as [?]: this means either your report.bib is not up-to-date or there is a spelling error in the label of the item you want to cite, either in report.bib or in report.tex}

\subsection{Formatting}

    \TODO{Incorrect number of keywords or HID and i523 not included in the keywords}
    \TODO{Other formatting issues}

\subsection{Writing Errors}

    \TODO{Errors in title, e.g. capitalization}
    \TODO{Spelling errors}
    \TODO{Are you using {\em a} and {\em the} properly?}
    \TODO{Do not use phrases such as {\em shown in the Figure below}. Instead, use {\em as shown in Figure 3}, when referring to the 3rd figure}
    \TODO{Do not use the word {\em I} instead use {\em we} even if you are the sole author}
    \TODO{Do not use the phrase {\em In this paper/report we show} instead use {\em We show}. It is not important if this is a paper or a report and does not need to be mentioned}
    \TODO{If you want to say {\em and} do not use {\em \&} but use the word {\em and}}
    \TODO{Use a space after . , : }
    \TODO{When using a section command, the section title is not written in all-caps as format does this for you}\begin{verbatim}\section{Introduction} and NOT \section{INTRODUCTION} \end{verbatim}

\subsection{Citation Issues and Plagiarism}

    \TODO{It is your responsibility to make sure no plagiarism occurs. The instructions and resources were given in the class}
    \TODO{Claims made without citations provided}
    \TODO{Need to paraphrase long quotations (whole sentences or longer)}
    \TODO{Need to quote directly cited material}

\subsection{Character Errors}

    \TODO{Erroneous use of quotation marks, i.e. use ``quotes'' , instead of " "}
    \TODO{To emphasize a word, use {\em emphasize} and not ``quote''}
    \TODO{When using the characters \& \# \% \_  put a backslash before them so that they show up correctly}
    \TODO{Pasting and copying from the Web often results in non-ASCII characters to be used in your text, please remove them and replace accordingly. This is the case for quotes, dashes and all the other special characters.}
    \TODO{If you see a figure and not a figure in text you copied from a text that has the fi combined as a single character}

\subsection{Structural Issues}

    \TODO{Acknowledgement section missing}
    \TODO{Incorrect README file}
    \TODO{In case of a class and if you do a multi-author paper, you need to add an appendix describing who did what in the paper}
    \TODO{The paper has less than 2 pages of text, i.e. excluding images, tables and figures}
    \TODO{The paper has more than 6 pages of text, i.e. excluding images, tables and figures}
    \TODO{Do not artificially inflate your paper if you are below the page limit}

\subsection{Details about the Figures and Tables}

    \TODO{Capitalization errors in referring to captions, e.g. Figure 1, Table 2}
    \TODO{Do use {\em label} and {\em ref} to automatically create figure numbers}
    \TODO{Wrong placement of figure caption. They should be on the bottom of the figure}
    \TODO{Wrong placement of table caption. They should be on the top of the table}
    \TODO{Images submitted incorrectly. They should be in native format, e.g. .graffle, .pptx, .png, .jpg}
    \TODO{Do not submit eps images. Instead, convert them to PDF}

    \TODO{The image files must be in a single directory named "images"}
    \TODO{In case there is a powerpoint in the submission, the image must be exported as PDF}
    \TODO{Make the figures large enough so we can read the details. If needed make the figure over two columns}
    \TODO{Do not worry about the figure placement if they are at a different location than you think. Figures are allowed to float. For this class, you should place all figures at the end of the report.}
    \TODO{In case you copied a figure from another paper you need to ask for copyright permission. In case of a class paper, you must include a reference to the original in the caption}
    \TODO{Remove any figure that is not referred to explicitly in the text (As shown in Figure ..)}
    \TODO{Do not use textwidth as a parameter for includegraphics}
    \TODO{Figures should be reasonably sized and often you just need to
  add columnwidth} e.g. \begin{verbatim}/includegraphics[width=\columnwidth]{images/myimage.pdf}\end{verbatim}

re

\end{document}
